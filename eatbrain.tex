\documentclass[10pt,a4paper]{article}
\usepackage[utf8]{inputenc}
\usepackage[italian]{babel}
\usepackage{subfiles}
\usepackage{titling}
\usepackage{graphicx}
\usepackage{epigraph}
\usepackage[bottom]{footmisc}
\renewcommand*{\thefootnote}{[\arabic{footnote}]}
\usepackage{hyperref}
\hypersetup{
	colorlinks,
	citecolor=black,
	filecolor=black,
	linkcolor=black,
	urlcolor=black
}
\newcommand{\URI}[3][blue]{\href{#2}{\color{#1}{#3}}}%

\setlength{\droptitle}{-10em}
\pretitle{%
	\begin{center}
		\LARGE
		\centering
		\includegraphics[width=10cm,height=10cm]{unipd}\\[\bigskipamount]
	}
\posttitle{
	\includegraphics[width=350px,height=140px]{logo}\\[\bigskipamount]
	\includegraphics[width=2cm,height=2cm]{logo2}\\[\bigskipamount]
\end{center}
}
\title{
  Eatbrain.net - Analisi Usabilità \\
  Università degli Studi di Padova \\
  Web Information Management
}
\author{
  Gianluca Marraffa, 1121080
}
\date{Giugno 2018}


\begin{document}

\maketitle
\thispagestyle{empty}
\newpage
\pagenumbering{roman}
\tableofcontents
\newpage
\pagenumbering{arabic}

\section{Introduzione e scopo del documento}
Al giorno d'oggi qualsiasi associazione piccola o grande dispone di un sito web per presentare la propria immagine al mondo. Lo sviluppo di un portale personale, spesso integrato con piattaforme sociali (i social network), è diventato infatti imperativo per promuovere in modo efficace la propria ragione d'essere. Col tempo le tecnologie si sono raffinate e da semplici documenti di ipertesto, le pagine web sono diventate delle vere e proprie applicazioni interattive, solo un fattore però era, è e sarà per sempre una costante: l'usabilità.
\setlength{\epigraphwidth}{0.8\textwidth}
\epigraph{Usability and user experience design is about designing products to be effective, efficient, and satisfying. Specifically, ISO defines usability as the “extent to which a product can be used by specified users to achieve specified goals effectively, efficiently and with satisfaction in a specified context of use”}{ISO 9241-11}
Questo documento si impegna ad analizzare l'usabilità desktop del sito web
\begin{center}
	\URI{https://eatbrain.net/}{eatbrain.net}
\end{center}
 evidenziandone pregi e difetti per poi concludere con una valutazione generale della fruizione dei contenuti.
\newpage

\section{Analisi Preliminare}
\subfile{parts/preliminare}
\section{Analisi Complessiva}
\label{sec-complessiva}
\subfile{parts/complessiva}
\section{Analisi Dettagliata}
In questa sezione effettueremo un analisi dettagliata delle pagine principali del portale, analizzando i 6 assi informativi e concludendo con una descrizione generale della pagina per individuare particolari pregi o difetti. \\
Partiremo dalla home page che merita la maggiore attenzione per poi proseguire con le altre in modo da verificare se sono state prese precauzioni per il fenomeno del deep linking\footnote{https://it.wikipedia.org/wiki/Deep\_linking}. \\
I 6 assi informativi altro non sono che la riposta alle 6 domande base\footnote{https://en.wikipedia.org/wiki/Five\_Ws} per ogni tipologia di raccolta di informazione, che sia essa nel giornalismo, nell'investigazione o nella scienza. \\
Nel caso dell'usabilità dei siti web possono essere così riassunte:
\begin{itemize}
	\item \textbf{Where}: In quale sito mi trovo? In quale parte del sito?
	\item \textbf{Who}: Chi si "nasconde" dietro questo sito?;
	\item \textbf{What}: Cosa mi offre questo sito, quali contenuti posso trovare?
	\item \textbf{Why}: Perché sono in questo sito? Quali benefici posso trarre?
	\item \textbf{How}: Come trovo le informazioni che cerco?
	\item \textbf{When}: Il sito è attivo? Quando è stato aggiornato l'ultima volta? Quali sono le ultime novità?
\end{itemize}
\subsection{Home Page}
\label{sec-home-page}
\subfile{parts/dettagliata/home}
\subsection{Deep Linking}
Considerata la struttura dei documenti web descritta nella sezione \hyperref[sec-struttura]{\texttt{Struttura}}, la gestione del deep linking nelle pagine interne e la relativa comprensione degli assi obbligatori (\textbf{Where}, \textbf{Who} e \textbf{What}) è stata abbondantemente descritta nella sezione dedicata alla \hyperref[sec-home-page]{\texttt{Home Page}} e rimane sostanzialmente invariata. \\
L'analisi delle altre pagine si concentrerà quindi sulle eventuali differenze che ci sono negli assi (tranne per il Why che rimane invariato), fornendo risposte per il contenuto interno. \\
Eventuali considerazioni potranno essere effettuate sulla propagazione degli errori descritti nella sezione \hyperref[sec-complessiva]{\texttt{Analisi Complessiva}}.
\subsection{Artists}
\label{sec-artists}
\subfile{parts/dettagliata/artists}
\subsection{Releases/Podcast}
\subfile{parts/dettagliata/releases}
\subsection{Events}
\subfile{parts/dettagliata/events}
\subsection{Altre Pagine}
L'analisi dettagliate termina qui, le altre pagine infatti non portano alcun contenuto significativo a questo documento: Contacts presenta informazioni estremamente limitate, Shop porta ad un altro portale web dedicato all'E-Commerce dell'etichetta mentre le pagine più interne seguono la struttura di quelle analizzate in precedenza, raccogliendo pregi e difetti già descritti nell'\hyperref[sec-complessiva]{\texttt{Analisi Complessiva}}.
\section{Valutazione e Considerazioni Finali}
Il sito effettua le sue mansioni in maniera efficace, ma detto in gergo comune: "Svolge il suo compitino e nulla di più". Questo è causato dall'assunzione che l'utente capitolato sul portale sia già a conoscenza dell'etichetta e che quindi non necessiti delle informazioni di base dell'attività (il maggiore lavoro di promozione è infatti effettuato sui canali social, intuibile dalla grande integrazione che viene fatta all'interno delle pagine). Questo purtroppo non rappresenta una scelta ottimale: per la struttura del web odierno non è improbabile che una persona venga reindirizzata al sito da fonti esterne. \\
Come analizzato in precedenza, il \textbf{What} rappresenta il punto forte dell'usabilità del sito, in ogni sezione viene infatti immediato comprendere cosa questa è in grado di offrire, tuttavia gli altri assi lasciano sempre un po' a desiderare sopratutto per quanto riguarda il tempo della loro individuazione. \\
L'\textbf{How} invece si pone come vero tallone d'Achille di tutto il portale:  senza avere una piccola conoscenza preliminare del contesto e della struttura del sito, può risultare difficile individuare immediatamente l'informazione ricercata. \\
La scelta del nome è ottima ma l'indicizzazzione nelle SERP risulta a malapena sufficiente, rendendo molto difficile raggiungere il sito tramite i motori di ricerca. \\
Nel complesso il sito offre un'usabilità sufficiente (principalmente causata dalla semplicità della struttura e dalla quantità non troppo elevata di informazione) ma presenta diverse inconsistenze che possono rendere negativa l'esperienza dell'utente meno esperto. \\
A partire dal rispetto delle convenzioni interne, basterebbero poche accortezze per migliorare la fruizione del sito ma il compito base viene svolto in modo incisivo. \\
\begin{center}
	\huge{\textbf{Voto: 6.5/10}}
\end{center}
\newpage
\appendix
\section{Tabelle delle figure}
\renewcommand{\arraystretch}{1.5}
\subsection{Immagini presenti nel documento}
In questa sezione vengono elencate le figure utilizzate nel documento, con relativo URL della pagina da cui sono state prese e nome del file. Le immagini qui elencate sono inserite nella cartella \texttt{immagini}. \\
Cliccando sul numero della figura si verrà reindirizzati a dove questa figura viene utilizzata (per motivi tecnici l'ancora è posta sul titolo della figura, una volta cliccato occore fare dello scroll verticale verso l'alto: -2 punti usabilità per il rapporto). \\
\begin{center}
\begin{tabular}{| c | c | c |}
	\hline
	\textbf{Numero figura} & \textbf{URL Pagina} & \textbf{Nome File} \\\hline
	\hyperref[struttura-default]{Figura 1} & N/A & struttura.png \\\hline
	\hyperref[menu]{Figura 2} & https://eatbrain.net/releases & menu.png \\\hline
	\hyperref[slogan-img]{Figura 3} & https://eatbrain.net/ & slogan.png \\\hline
	\hyperref[soundcloud-integration]{Figura 4} & https://eatbrain.net/eatbrain-060 & soundcloud.png \\\hline
	\hyperref[non-link]{Figura 5} & https://eatbrain.net/eatbrain-060 & non-link.png \\\hline
	\hyperref[page-404]{Figura 6} & https://eatbrain.net/pippo & 404.png \\\hline
	\hyperref[page-500]{Figura 7} & https://eatbrain.net/releases/ & 500.png \\\hline
	\hyperref[page-404-bad]{Figura 8} & https://eatbrain.net/pippo/ & 404-bad.png \\\hline
	\hyperref[home-visible]{Figura 9} & https://eatbrain.net/ & home-visible.png \\\hline
	\hyperref[home-where]{Figura 10} & https://eatbrain.net/ & home-why.png \\\hline
	\hyperref[img-home-who]{Figura 11} & https://eatbrain.net/ & home-who-2.png \\\hline
	\hyperref[logos-everywhere]{Figura 12} & https://eatbrain.net/ & logos-everywhere.png \\\hline
	\hyperref[home-why]{Figura 13} & https://eatbrain.net/ & slogan.png \\\hline
	\hyperref[img-home-what]{Figura 14} & https://eatbrain.net/ & home-what.png \\\hline
	\hyperref[img-home-how]{Figura 15} & https://eatbrain.net/ & home-how.png \\\hline
	\hyperref[home-when-1]{Figura 16} & https://eatbrain.net/ & home-when-1.png \\\hline
	\hyperref[home-when-2]{Figura 17} & https://eatbrain.net/ & home-when-2.png \\\hline
	\hyperref[img-artists]{Figura 18} & https://eatbrain.net/artists & artists.png \\\hline
	\hyperref[img-releases]{Figura 19} & https://eatbrain.net/releases & releases.png \\\hline
	\hyperref[img-events]{Figura 20} & https://eatbrain.net/events & events.png \\\hline
\end{tabular}
\end{center}
\newpage
\subsection{Pagine Utilizzate}
In questa sezione vengono elencate gli snapshot delle pagine analizzate durante l'analisi del sito. Le immagini qui elencate sono inserite nella cartella \texttt{pagine}.\\
\begin{center}
\setlength{\tabcolsep}{15pt}
\begin{tabular}{| c | c |}
	\hline
	\textbf{URL Pagina}  & \textbf{Nome File} \\\hline
	https://eatbrain.net/ & home.png \\\hline
	https://eatbrain.net/ (fine pagina) & home-footer.png \\\hline
	https://eatbrain.net/artists & artists.png \\\hline
	https://eatbrain.net/artists (fine pagina) & artists-2.png \\\hline
	https://eatbrain.net/releases & releases.png \\\hline
	https://eatbrain.net/podcasts & podcasts.png \\\hline
	https://eatbrain.net/events & events.png \\\hline
	https://eatbrain.net/contacts & contacts.png \\\hline
	https://eatbrain.net/eatbrain-060 & dinosaur.png \\\hline
\end{tabular}
\end{center}

\end{document}