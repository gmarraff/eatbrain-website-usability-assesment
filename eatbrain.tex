\documentclass[10pt,a4paper]{article}
\usepackage[utf8]{inputenc}
\usepackage[italian]{babel}
\usepackage{subfiles}
\usepackage{titling}
\usepackage{graphicx}
\usepackage{epigraph}
\usepackage[bottom]{footmisc}
\renewcommand*{\thefootnote}{[\arabic{footnote}]}
\usepackage{hyperref}
\hypersetup{
	colorlinks,
	citecolor=black,
	filecolor=black,
	linkcolor=black,
	urlcolor=black
}
\newcommand{\URI}[3][blue]{\href{#2}{\color{#1}{#3}}}%

\setlength{\droptitle}{-10em}
\pretitle{%
	\begin{center}
		\LARGE
		\centering
		\includegraphics[width=10cm,height=10cm]{unipd}\\[\bigskipamount]
	}
\posttitle{
	\includegraphics[width=350px,height=140px]{logo}\\[\bigskipamount]
	\includegraphics[width=2cm,height=2cm]{logo2}\\[\bigskipamount]
\end{center}
}
\title{
  Eatbrain.net - Analisi Usabilità \\
  Università degli Studi di Padova \\
  Web Information Management
}
\author{
  Gianluca Marraffa, 1121080
}
\date{Giugno 2018}


\begin{document}

\maketitle
\thispagestyle{empty}
\newpage
\pagenumbering{roman}
\tableofcontents
\newpage
\pagenumbering{arabic}

\section{Introduzione e scopo del documento}
Al giorno d'oggi qualsiasi associazione piccola o grande dispone di un sito web per presentare la propria immagine al mondo. Lo sviluppo di un portale personale, spesso integrato con piattaforme sociali (i social network), è diventato infatti imperativo per promuovere in modo efficace la propria ragione d'essere. Col tempo le tecnologie si sono raffinate e da semplici documenti di ipertesto, le pagine web sono diventate delle vere e proprie applicazioni interattive, solo un fattore però era, è e sarà per sempre una costante: l'usabilità.
\setlength{\epigraphwidth}{0.8\textwidth}
\epigraph{Usability and user experience design is about designing products to be effective, efficient, and satisfying. Specifically, ISO defines usability as the “extent to which a product can be used by specified users to achieve specified goals effectively, efficiently and with satisfaction in a specified context of use”}{ISO 9241-11}
Questo documento si impegna ad analizzare l'usabilità desktop del sito web
\begin{center}
	\URI{https://eatbrain.net/}{eatbrain.net}
\end{center}
 evidenziandone pregi e difetti per poi concludere con una valutazione generale della fruizione dei contenuti.
\newpage

\section{Analisi Preliminare}
\subfile{parts/preliminare}
\section{Analisi Complessiva}
\label{sec-complessiva}
\subfile{parts/complessiva}
\section{Analisi Dettagliata}
In questa sezione effettueremo un analisi dettagliata delle pagine principali del portale, analizzando i 6 assi informativi e concludendo con una descrizione generale della pagina per individuare particolari pregi o difetti. \\
Partiremo dalla home page che merita la maggiore attenzione per poi proseguire con le altre in modo da verificare se sono state prese precauzioni per il fenomeno del deep linking\footnote{https://it.wikipedia.org/wiki/Deep\_linking}. \\
I 6 assi informativi altro non sono che la riposta alle 6 domande base\footnote{https://en.wikipedia.org/wiki/Five\_Ws} per ogni tipologia di raccolta di informazione, che sia essa nel giornalismo, nell'investigazione o nella scienza. \\
Nel caso dell'usabilità dei siti web possono essere così riassunte:
\begin{itemize}
	\item \textbf{Where}: In quale sito mi trovo? In quale parte del sito?
	\item \textbf{Who}: Chi si "nasconde" dietro questo sito?;
	\item \textbf{What}: Cosa mi offre questo sito, quali contenuti posso trovare?
	\item \textbf{Why}: Perché sono in questo sito? Quali benefici posso trarre?
	\item \textbf{How}: Come trovo le informazioni che cerco?
	\item \textbf{When}: Il sito è attivo? Quando è stato aggiornato l'ultima volta? Quali sono le ultime novità?
\end{itemize}
\subsection{Home Page}
\label{sec-home-page}
\subfile{parts/dettagliata/home}
\subsection{Deep Linking}
Considerata la struttura dei documenti web descritta nella sezione \hyperref[sec-struttura]{\texttt{Struttura}}, la gestione del deep linking nelle pagine interne e la relativa comprensione degli assi obbligatori (\textbf{Where}, \textbf{Who} e \textbf{What}) è stata abbondantemente descritta nella sezione dedicata alla \hyperref[sec-home-page]{\texttt{Home Page}} e rimane sostanzialmente invariata. \\
L'analisi delle altre pagine si concentrerà quindi sulle eventuali differenze che ci sono negli assi (tranne per il Why che rimane invariato), fornendo risposte per il contenuto interno. \\
Eventuali considerazioni potranno essere effettuate sulla propagazione degli errori descritti nella sezione \hyperref[sec-complessiva]{\texttt{Analisi Complessiva}}.
\subsection{Artists}
\label{sec-artists}
\subfile{parts/dettagliata/artists}
\subsection{Releases/Podcast}
\subfile{parts/dettagliata/releases}
\subsection{Events}
\subfile{parts/dettagliata/events}
\subsection{Altre Pagine}
L'analisi dettagliate termina qui, le altre pagine infatti non portano alcun contenuto significativo a questo documento: Contacts presenta informazioni estremamente limitate, Shop porta ad un altro portale web dedicato all'E-Commerce dell'etichetta mentre le pagine più interne seguono la struttura di quelle analizzate in precedenza, raccogliendo pregi e difetti già descritti nell'\hyperref[sec-complessiva]{\texttt{Analisi Complessiva}}.
\section{Valutazione e Considerazioni Finali}
Voto Dominio 9.5 \\
Voto SEO 6 \\
Soffre dell'assunzione che chi arriva alla pagina conosce già di cosa di parla
Buona navigazione conseguenza della semplicità del sito
Offre diversi problemi di inconsistenza
Nel complessivo ha un'usabilità sufficiente (principalmente per la semplicità della pagina), ma potrebbe essere fatto molto di più
Voto finale 6.5 oppure 7 devo decidere \\
Ottimo il what ma gli altri assi lasciano sempre un po' a desiderare a causa dell'assunzione che l'utente segua già i canali social
\appendix
\section{Tabella delle figure}

\end{document}