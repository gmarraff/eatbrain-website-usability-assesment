\documentclass[10pt,a4paper]{article}
\usepackage[utf8]{inputenc}
\usepackage[italian]{babel}
\usepackage{subfiles}
\usepackage{titling}
\usepackage{graphicx}
\usepackage{epigraph}
\usepackage{hyperref}
\hypersetup{
	colorlinks,
	citecolor=black,
	filecolor=black,
	linkcolor=black,
	urlcolor=black
}
\newcommand{\URI}[3][blue]{\href{#2}{\color{#1}{#3}}}%

\setlength{\droptitle}{-10em}
\pretitle{%
	\begin{center}
		\LARGE
		\centering
		\includegraphics[width=10cm,height=10cm]{unipd}\\[\bigskipamount]
	}
\posttitle{
	\includegraphics[width=350px,height=140px]{logo}\\[\bigskipamount]
	\includegraphics[width=2cm,height=2cm]{logo2}\\[\bigskipamount]
\end{center}
}
\title{
  Eatbrain.net - Analisi Usabilità \\
  Università degli Studi di Padova \\
  Web Information Management
}
\author{
  Gianluca Marraffa, 1121080
}
\date{Primo Semestre, A.A 2017/18}


\begin{document}

\maketitle
\newpage
\pagenumbering{roman}
\tableofcontents
\newpage
\pagenumbering{arabic}

\section{Introduzione e scopo del documento}
Al giorno d'oggi qualsiasi associazione piccola o grande dispone di un sito web per presentare la propria immagine al mondo. Lo sviluppo di un portale personale, spesso integrato con piattaforme sociali (i social network), è diventato infatti imperativo per promuovere in modo efficacie la propria ragione d'essere. Col tempo le tecnologie si sono raffinate e da semplici documenti di ipertesto, le pagine web sono diventate delle vere e proprie applicazioni interattive, solo un fattore però era, è e sarà per sempre una costante: l'usabilità.
\setlength{\epigraphwidth}{0.8\textwidth}
\epigraph{Usability and user experience design is about designing products to be effective, efficient, and satisfying. Specifically, ISO defines usability as the “extent to which a product can be used by specified users to achieve specified goals effectively, efficiently and with satisfaction in a specified context of use”}{ISO 9241-11}
Questo documento si impegna ad analizzare l'usabilità del dominio
\begin{center}
	\URI{https://eatbrain.net/}{eatbrain.net}
\end{center}
 evidenziandone pregi e difetti per poi concludere con una valutazione generale della fruizione dei contenuti.
 \newpage
\section{Analisi Preliminare}
\subsection{Eatbrain}
\setlength{\epigraphwidth}{0.5\textwidth}
\epigraph{Dark. Deep. Unfiltered. Unrestrained. This is premier neurofunk drum \& bass. This is Eatbrain.}{Eatbrain}
Eatbrain è un'etichetta discografica nata del 2011 dalla mente di Gabor Simon, altrimenti conosciuto come Jade: dj e producer pioniere del genere musicale Neurofunk. \\
La Neurofunk è un sotto genere della Drum'n'Bass, genere di musica elettronica nato negli anni '90 nel Regno Unito. Eatbrain quindi si impegna a produrre e distribuire musica di vari artisti nel panorama internazionale, principlmente europeo. \\
Essendo un genere musicale abbastanza di nicchia, i visitatori del sito con molta probabilità sono già a conoscenza del lavoro svolto dall'etichetta, ma questo non può essere una scusante per eventuali lacune sul lato informativo: visitatori capitati per caso sul sito (ad esempio reindirizzati da negozi musicali virtuali o servizi di streaming audio) devo poter capire cosa stanno guardando. \\
L'obiettivo del sito è quello di aggiornare gli ascoltatori sulle attività dell'etichetta e promuovere i membri che ne fanno parte ma non solo: è possibile scorrere lo storico delle passate produzioni musicali e ascoltare le tracce che hanno reso celebri l'etichetta nel mondo della musica Neurofunk. 
\subsection{Dominio}
\subsection{SEO}
\section{Analisi Complessiva}
\subsection{Struttura}
\subsection{Le 6 W}
\subsection{Integrazione con i Social Media}
\subsection{Pubblicità}
\section{Home Page}
\section{Artists}
\section{Releases/Podcast}
\section{Events}
\section{Contacts}
\section{Altre Pagine}
\section{Valutazione e Considerazioni Finali}
\appendix
\section{Tabella delle figure}

\end{document}