\documentclass[../eatbrain.tex]{subfile}

\subsection{Eatbrain}
\setlength{\epigraphwidth}{0.5\textwidth}
\epigraph{Dark. Deep. Unfiltered. Unrestrained. This is premier neurofunk drum \& bass. This is Eatbrain.}{Eatbrain}
Eatbrain è un'etichetta discografica nata del 2011 dalla mente di Gabor Simon, altrimenti conosciuto come Jade: dj e producer pioniere del genere musicale Neurofunk. \\
La Neurofunk è un sotto genere della Drum'n'Bass, genere di musica elettronica nato negli anni '90 nel Regno Unito. Eatbrain quindi si impegna a produrre e distribuire musica di vari artisti nel panorama internazionale, principlmente europeo. \\
Essendo un genere musicale abbastanza di nicchia, i visitatori del sito con molta probabilità sono già a conoscenza del lavoro svolto dall'etichetta, ma questo non può essere una scusante per eventuali lacune sul lato informativo: visitatori capitati per caso sul sito (ad esempio reindirizzati da negozi musicali virtuali o servizi di streaming audio) devo poter capire cosa stanno guardando. \\
L'obiettivo del sito è quello di aggiornare gli ascoltatori sulle attività dell'etichetta e promuovere i membri che ne fanno parte ma non solo: è possibile scorrere lo storico delle passate produzioni musicali e ascoltare le tracce che hanno reso celebri l'etichetta nel mondo della musica Neurofunk. 
\subsection{Dominio}
Il nome del sito (\textbf{Eatbrain}) risulta molto efficace, volontariamente o no infatti rispetta quasi tutte le regole d'oro per la selezione del nome:
\begin{itemize}
	\item Il nome ha una lunghezza media, e rimane facile da pronunciare, inoltre non usa parole inventate ma è la combinazione tra due: Eat (Mangiare) e Brain (Cervello);
	\item Il nome inizia con una vocale, massimizzando l'impatto della sua pronuncia e non utilizza separatori per dividere le due parole utilizzate;
	\item Nonostante la frase abbia un connotato negativo (mangiare cervelli), risulta molto adeguata per il prodotto che propone: la musica neurofunk spesso fa da colonna sonora a scenari apocalittici/post-industriali;
	\item L'unica pecca si può trovare nella scelta del dominio, il \texttt{.net} si impone di meno rispetto al \texttt{.com}, ma nell'insieme generale questa scelta non penalizza troppo l'efficacia del dominio.
\end{itemize}
\subsection{SEO}
Il lavoro effettuato per ottimizzare la presenza nei motori di ricerca risulta scadente: analizzando le keyword \texttt{eatbrain, neurofunk, drum n bass, bass music} e \texttt{dnb} ottieniamo i seguenti risultati:
\begin{itemize}
	\item \textbf{eatbrain}: primo posto;
	\item \textbf{neurofunk}: ottavo posto;
	\item \textbf{drum n bass}: fuori dalla prima pagina;
	\item \textbf{dnb}: fuori dalla prima pagina;
	\item \textbf{bass music}: fuori dalla prima pagina;
\end{itemize}
Per un utente che muove i primi passi verso questo genere musicale viene molto difficile trovare tramite i motori di ricerca la casa discografica. \\
Risultati migliori si ottengono associando la parola \texttt{neurofunk} agli aggettivi \texttt{premier, unfiltered, unstrained}, presenti nella descrizione dell'etichetta discografica, ma per un utente che ancora non la conosce questo non può essere d'aiuto. \\
Con il termine \texttt{neurofunk label} invece il sito web appare secondo nelle SERP, avendo un vantaggio sulle concorrenti (al primo posto infatti troviamo un post su reddit).
